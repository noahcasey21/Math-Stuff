\documentclass[10pt,a4paper]{report}
\usepackage[utf8]{inputenc}
\usepackage{amsmath}
\usepackage{amsfonts}
\usepackage{amssymb}
\usepackage{makeidx}
\usepackage{setspace}
\usepackage{titlesec}
\usepackage{tikz}
\usepackage{xparse}
\usetikzlibrary{positioning}

\renewcommand\thesubsection{\alph{subsection}}
\renewcommand\thesection{\arabic{section}}
\titleformat{\subsection}[runin]{\normalfont\large\bfseries}{\thesubsection}{1em}{}
\titleformat{\section}[runin]{\normalfont\Large\bfseries}{\thesection}{1em}{}

\ExplSyntaxOn
\NewDocumentCommand{\cycle}{ O{\;} m }
 {
  (
  \alec_cycle:nn { #1 } { #2 }
  )
 }

\seq_new:N \l_alec_cycle_seq
\cs_new_protected:Npn \alec_cycle:nn #1 #2
 {
  \seq_set_split:Nnn \l_alec_cycle_seq { , } { #2 }
  \seq_use:Nn \l_alec_cycle_seq { #1 }
 }
\ExplSyntaxOff

\doublespacing
\begin{document}

\begin{center}
\textbf{Noah Casey}\\
\textbf{MATH 4720}\\
\textbf{Homework 3}
\end{center}

\section{}
\underline{For the group $\mathbb{Z}_{225}$, using bar notation, e.g. $\bar{5}$ for the element in $\mathbb{Z}_{225}$}:

\subsection{}
\underline{Draw a subgroup diagram. Cite the theorem that allows for this to be done easily.}\newline
\indent The theorem that allows for easy diagramming of subgroups (particularly in $\mathbb{Z}_{225}$) is Theorem G9 (ii), which states that for a cyclic group of order $n$, there is a one-to-one correspondence between the positive divisors of $n$ and the set of subgroups. Hence, the positive divisors of 225 correspond to the subgroups of $\mathbb{Z}_{225}$ as the group is cyclic and of order 225.
    
    % center everything in the figure
    \begin{center}
    % horizontal node distance
    \newcommand{\mydistance}{.6cm}
    \begin{tikzpicture}[node distance=1cm]
    \title{Subgroups of $\mathbb{Z}_{225}$}
    \node(1)                           {$\bar{1} = \mathbb{Z}_{225}$};
    \node(3)       [below left=1cm and 1cm of 1] {$\bar{3}$};
    \node(5)      [below right=1cm and 1cm of 1]  {$\bar{5}$};
    \node(9)      [below left=1cm and 1cm of 3]       {$\bar{9}$};
    \node(15)      [below right=1cm and 1cm of 3]       {$\bar{15}$};
    \node(25)      [below right=1cm and 1cm of 5]       {$\bar{25}$};
    \node(45)      [below right=1cm and 1cm of 9]       {$\bar{45}$};
    \node(75)      [below right=1cm and 1cm of 15]       {$\bar{75}$};
    \node(225)      [below right=1cm and 1cm of 45]      {$\bar{225} = \bar{0}$};

    %draw lines
    \foreach \x in {3, 5}
    {
        \draw (1) -- (\x);
    }
    \foreach \x in {9, 15}
    {
        \draw (3) -- (\x);
    }
    \foreach \x in {15, 25}
    {
        \draw (5) -- (\x);
    }
    \draw (9) -- (45);
    \draw (15) -- (75);
    \draw (25) -- (75);
    \foreach \x in {45, 75}
    {
        \draw (225) -- (\x);
    }
    \end{tikzpicture}
    \end{center}

\noindent\subsection{}
\underline{How many subgroups of order 15 are there? Explain briefly.}\newline
\indent By Theorem G9 (ii), if $k$ is a divisor of 225, then there is a unique subgroup of order $\frac{225}{k}$. If $k=15$, then the order of the subgroup generated by $k$ is 15. Thus, there is a single subgroup that has order 15.

\noindent\subsection{}
\underline{Find all of the elements that generate a subgroup of order 15.}\newline
\indent By Theorem G9 (i), $ka$ (additive notaion) is a generator of a group of order n if and only if $\gcd(k, n)=1$. Hence, the $k$-values that generate a subgroup of order 15 must be relatively prime to 15. These are $k=\{1,2,3,4,7,8,11,13,14\}$, hence, as (we know $a=15$ as $\bar{15} is a generator of the subgroup of order 15 from (b), $k15=\{15,30,45,60,105,120,165,195,210\}$ are the generators of the subgroup of order 15.

\section{}
\underline{Let $G$ be a group with $a\in G$. Show that $o(a)=o(a^{-1})$}:\newline
\indent Let $G$ be finite, then $o(a)=n$ if and only if $a^{n}=e$, so $(a^{-1})^{n}=a^{-n}=(a^{n})^{-1}=e^{-1}=e$ and $o(a^{-1})=n$. \newline
\indent Now let $G$ be an infinite group. Note that $o(a)=\infty$, so $a^{n}\neq e$ for any $n\in\mathbb{N}$. Now assume that $o(a^{-1})=n$ for some $n\in\mathbb{N}$. This may be the case only if $a^{n}=e$ or $a^{-1}=e$. We know that the former is false as $o(a)=\infty$, so for the statment to be true, it must be the case that $a^{-1}=e$. However, $ae=a$, so this is false. Hence, $o(a^{-1})=\infty$.

\section{}
\underline{Show that $S_{n}=\langle\cycle{1,2},\cycle{1,3},...,\cycle{1,n}\rangle$.}\newline
\indent We proceed by induction on arbitrary $m$-cycles. \newline
Base Cases: let $m=2$ such that we have the cycle $\cycle{a_{1}, a_{2}}$, then this can be written as $\cycle{1,a_{1}}\cycle{1,a_{2}\cycle{1,a_{1}}}$. Now let $m=3$, then some cycle $\cycle{a_{1},a_{2},a_{3}}$ may be written as $\cycle{1,a_{3}}\cycle{1,a_{2}\cycle{1,a_{1}}\cycle{1,a_{3}}}$. Both of these cases give cycles as products of transpositions of the form desired.\newline
Induction step: Now assume that some $m$-cycle may be written as 
\[
\cycle{1,a_{m}}\cycle{1,a_{m-1}}\dots\cycle{1,a_{2}}\cycle{1,a_{1}}\cycle{1,a_{m}}}.
\]
 It remains to show that some $m+1$-cycle can be written in a similar fashion. All we need to do is extend the $m$-cycle to account for this extra $a_{m+1}$. We know that $a_{m+1}$ must be mapped to $a_{1}$, so we can replace the rightmost $a_{m}$ with $a_{m+1}$. Now, we need $a_{m}$ to map to $a_{m+1}$, so we append an $a_{m+1}$ to the left of $a_{m}$ in the cycle. The rest of the mappings are uncahnging between the two cycles, so the $m+1$-cycle may be written as 
 \[
 \cycle{1,a_{m+1}}\cycle{1,a_{m}}\cycle{1,a_{m-1}}\dots\cycle{1,a_{2}}\cycle{1,a_{1}}\cycle{1,a_{m+1}}}.
 \]

\section{}
\underline{Let $\tau = \cycle{a_{1},a_{2},...,a_{k}}$ be a k-cycle in $S_{n}$.}\newline

\subsection{}
\underline{Prove that if $\sigma$ is any permutation in $S_{n}$, then $\sigma\tau\sigma^{-1}=\cycle{\sigma(a_{1}),\sigma(a_{2}),...,\sigma(a_{k})}$.}\newline
\indent Set $\{\sigma(a_{1}),\sigma(a_{2}),...,\sigma(a_{k})\}\in\{1,...,n\}$. Then 
\[
    \sigma\tau\sigma^{-1}(\sigma(a_{1}))=\sigma\tau(a_{1})=\sigma(a_{2}).
\]
Now take some $a_{i}\in\{a_{1},a_{2},...,a_{k}\}$. We have that 
\[
\sigma\tau\sigma^{-1}(\sigma(a_{i}))=\sigma\tau(a_{i})=\sigma(a_{i+1}).
\]
If $i=k$, then $i+1=1$ as $\tau$ maps $a_{k}$ to $a_{1}$. Now take some $m\notin\{a_{1},a_{2},...,a_{k}\}$. Note that $\sigma\tau\sigma^{-1}(\sigma(m))=\sigma\tau(m)=\sigma(m)$ as $\tau$ fixes $m$, so $\sigma\tau\sigma^{-1}$ fixes $\sigma(m)$.

\subsection{}
\underline{Let $\mu$ be a k-cycle. Prove that there is a permutation $\sigma$ such that $\sigma\tau\sigma^{-1}=\mu$.}\newline
\indent Assume that $\mu = \cycle{\mu_{1},\mu_{2},...,\mu_{k}}$ where $\{\mu_{1},\mu_{2},...,\mu_{k}\}\in\{1,...,n\}$. As we know from $(a)$, $\sigma\tau\sigma^{-1}=\cycle{\sigma(a_{1}),\sigma(a_{2}),...,\sigma(a_{k})}$; thus if we set $\sigma(a_{i})=\mu_{i}$ and fix all letters not in $\{a_{1},...,a_{k}\}$, then we have found a $\sigma$ that satisfies the condition. 
\newline\indent More precisely, we may set $\sigma=\cycle{a_{1},\mu_{1}}\cycle{a_{2},\mu_{2}}\dots\cycle{a_{k},\mu_{k}}$. This function takes any $a_{i}$ and maps it to $\mu_{i}$. 
We now handle the case in which we have cycles in the product of the form $\cycle{a_{i}=\mu_{j},\mu_{i}}\dots\cycle{a_{j},\mu_{j}}$. Note that if this is the case, then $a_{j}$ is sent to $\mu_{i}$, which is undesirable. However, if we flip the order of these transpositions to $\cycle{a_{j},\mu_{j}}\dots\cycle{a_{i}=\mu_{j},\mu_{i}}$, then we see that $a_{i}$ is mapped to $\mu_{i}$ and $a_{j}$ is mapped to $\mu_{j}$ as desired. Flipping the order of any cycles of this form gives the desired $\sigma$ as 
\[
\mu(\sigma(a_{i}))=\mu(\mu_{i})=\mu_{i+1}
\]
and 
\[
\sigma\tau\sigma^{-1}(\sigma(a_{i}))=\sigma\tau(a_{i})=\sigma(a_{i+1})=\mu_{i+1}.
\]

\section{}
\underline{Find all of the left and right cosets of $H=\langle\cycle{1,2,3}\rangle$ in $S_{4}$.}\newline\underline{If you had to find the left and right cosets in $S_{5}$, how many of each would there be?}\newline
\indent We see that $H=\{\cycle{1,2,3},\cycle{1,3,2},id\}$, so the cosets are as follows.\newline
\begin{tabular}{c | c}
    \textbf{Left cosets} & \textbf{Right cosets}\\
    $idH=H$ & $Hid=H$\\
    $\cycle{1,2}H=\{\cycle{1,2},\cycle{2,3},\cycle{1,3}\}$ & $H\cycle{1,2}=\{\cycle{1,2},\cycle{2,3},\cycle{1,3}\}$\\
    $\cycle{1,4}H=\{\cycle{1,4},\cycle{1,2,3,4},\cycle{1,3,2,4}\}$ & $H\cycle{1,4}=\{\cycle{1,4},\cycle{1,4,2,3},\cycle{1,4,3,2}\}$\\
    $\cycle{2,4}H=\{\cycle{2,4},\cycle{1,4,2,3},\cycle{1,3,4,2}\}$ & $H\cycle{2,4}=\{\cycle{2,4},\cycle{1,2,4,3},\cycle{1,3,2,4}\}$\\
    $\cycle{3,4}H=\{\cycle{3,4},\cycle{1,2,4,3},\cycle{1,4,3,2}\}$ & $H\cycle{3,4}=\{\cycle{3,4},\cycle{1,2,3,4},\cycle{1,3,4,2}\}$\\
    $\cycle{1,2,4}H=\{\cycle{1,2,4},\cycle{1,4}\cycle{2,3},\cycle{1,3,4}\}$ & $H\cycle{1,2,4}=\{\cycle{1,2,4},\cycle{1,3}\cycle{2,4},\cycle{2,4,3}\}$\\
    $\cycle{1,4,2}H=\{\cycle{1,4,2},\cycle{2,3,4},\cycle{1,3}\cycle{2,4}\}$ & $H\cycle{1,4,2}=\{\cycle{1,4,2},\cycle{1,4,3},\cycle{1,4}\cycle{2,3}\}$\\
    $\cycle{1,4,3}H=\{\cycle{1,4,3},\cycle{1,2}\cycle{3,4},\cycle{2,4,3}\}$ & $H\cycle{1,3,4}=\{\cycle{1,3,4},\cycle{2,3,4},\cycle{1,2}\cycle{3,4}\}$\\
\end{tabular} 

Also note that each coset $aH=bH=cH=\{a,b,c\}$, so each coset above may be relabeled using any of the elements in the set.\newline
\indent In $S_{5}$, there are $5!$ elements. Each coset of $H$ has a size of 3, and as cosets are equivalence relations, they parition the set. Hence, there will be $\frac{5!}{3}=40$ left cosets of $H$ in $S_{5}$; likewise, there will be 40 right cosets.

\section{}
\underline{Do the computations without need for a calculator}\newline

\subsection{}
\underline{Use FLT to "primality test" the number 35 for primeness}\newline
\indent Note that 35 does not divide 6, so we may use $a=6$ as specified in Fermat's Little theorem. If 35 is a candidate for a prime number, then $6^{34}\equiv1\pmod{35}$. 
We see that $6^{2}=36\equiv1\pmod{35}$, so $6^{34}\equiv(6^{2})^{17}\equiv1^{17}\equiv1\pmod{35}$. Thus, we cannot claim that 35 is not prime based on FLT. However, the implication only claims that if 35 is prime, then $6^{34}\equiv1\pmod{35}$, but the implication doesn't go the other way, so we cannot say with certainty that 35 is prime (in fact, one can see that it is not).

\subsection{}
\underline{Verify that Euler's theorem holds for $n=35$ and $a=2$}\newline
\indent Clearly, $\gcd(35,2)=1$ as 2 does not divide 35. Thus, by Euler's theorem, $2^{\varphi(35)}\equiv1\pmod{35}$. We verify this by first determining $\varphi(35)$. Recall that the Euler-phi function with $n=35$ is defined as $\varphi(35)=|\{k|\gcd(k,35)=0\}|$ where $1\leq k<35$. 
The set of numbers less than 35 that are relatively prime with 35 are $\{1,2,3,4,6,8,9,11,12,13,16,17,18,19,22,23,24,26,27,29,31,32,33,34\}$ and the size of this set is 24. Thus, $\varphi(35)=24$ and it must be the case that $2^{24}\equiv1\pmod{35}$. 
We verify this by noting that $2^{24}\equiv(2^{3})^{8}\equiv(8)^{8}\equiv(8^{2})^{4}\equiv64^{4}\equiv29^{4}\equiv(-6)^{4}\equiv36^{2}\equiv1^{2}\equiv1\pmod{35}$. Thus, Euler's theorem holds for $n=35$ and $a=2$.

\section
\underline{Let $G$ be a finite group. Suppose $G$ has subgroups of order 8, 90, and 220. What can you say about the order of $G$? }\newline
\indent By Corollary G14 (Lagrange's Theorem), the order of subgroup $H$ of $G$ divides the order of $G$. So, we can say that $|G|$ is a multiple of 8, 90, and 220. Furthermore, we may find the minimum order of $G$. Let $|G|=n$. As $90\vert n$, $n\geq 90$. However, $8\nmid 90$, but 8 and 90 both divide 180, and this is the smallest number that both divide. Now since $220\vert n$, $n\geq 220$, but $180\nmid 220$. Thus, we find the smalles number that both divide; note that both divide $(18)(22)(10)$ as $180\vert (180)(22)$ and $220\vert (18)(220)$. Thus, $|G|=(k18)(l22)(n10)$ for some $k,l,n\in\mathbb{N}$.

\centerline{\textbf{Sources}}
\begin{itemize}
    \item Python: for confirming numerical computations.
    \item Stack Exchange: understanding why a permutation can be written as a product of transpositions.
\end{itemize}


\end{document}